
%8%%8%%8%%8%%8%%8%%8%%8%%8%%8%%8%%8%%8%%8%%8%%8%%8%%8%%8
%                      Conclusão                       %
%8%%8%%8%%8%%8%%8%%8%%8%%8%%8%%8%%8%%8%%8%%8%%8%%8%%8%%8

\chapter{Conclusão}

\label{chap:conclusion}

Durante esta pesquisa, abordamos a relevância e a aplicação das soft skills para a qualificação de um profissional. Com foco no papel do programador, profissão da área de desenvolvimento de software e de Tecnologia da Informação, fizemos o levantamento de importantes soft skills que o respectivo setor de trabalho exige. Destacamos as soft skills Análise e resolução de problemas, Atenção a detalhes, Aprendizagem rápida, Persistência, Comunicação e Trabalho independente como habilidades diferenciais para integração, permanência e crescimento do programador como profissional.

Através de conceitos provenientes de teorias sobre personalidade, como o Modelo dos Cinco Fatores (FFM), buscamos esclarecer o significado de cada soft skill em destaque, aprendendo as características e comportamento dos indivíduos que as possuem. Discutimos também o motivo pelo qual elas são importantes para o programador de software e como aplicam-se no contexto dessa profissão. Assim, construímos uma base teórica e conceitual para o desenvolvimento de uma estratégia que visa oferecer uma alternativa para identificação de soft skills em programadores, utilizando-se de uma abordagem automática.

Nossa estratégia propõe um conjunto de sete métricas que pontuam o nível das soft skills que um indivíduo possui. As métricas são aplicadas a um sistema de programação online chamado Huxley, categorizado como juiz online. A partir da base de dados do Huxley, coletamos essas métricas e elaboramos um estudo empírico para validação das mesmas. Esse estudo também utiliza-se do FFM, através de um questionário a respeito de personalidade chamado TIPI (Ten-Item Personality Inventory). Os participantes do estudo, como usuários do Huxley, foram avaliados diante das métricas que identificam suas soft skills. Os mesmos também responderam o TIPI e comparamos os resultados em busca de analisar se as métricas das soft skills condizem com os traços de personalidades.

De acordo com os resultados de nosso estudo de validação, as métricas Comunicação e Trabalho independente não mostram resultados relevantes. Suas correlações com os traços de personalidades esperados não são significantes. 
A métrica de Comunicação considera apenas comentários no código-fonte. No entanto, a simples contagem por comentários, sem primeiro analisá-los, e ainda a falta de outras funcionalidades de comunicação que poderiam ser exploradas fez com que a métrica não expressasse a respectiva soft skill.
Já a métrica de Trabalho independente considera os pedidos de ajuda e dicas. Nesse caso, levando em conta que os usuários do juiz online utilizado estou em fase de aprendizagem, torna-se normal que os mesmos se utilizem desses recursos para retirar dúvidas, o que possivelmente interferiu no resultado da métrica.
Sendo assim, indicamos que ambas métricas necessitam ser revistas e aprimoradas. Não aconselhamos a utilização das mesmas para identificação automática de soft skills no contexto do Huxley.

Por outro lado, podemos indicar cinco métricas que funcionam de maneira satisfatória, são elas: Análise de problemas, Resolução de problemas, Atenção a detalhes, Aprendizagem rápida e Persistência. Essas métricas apresentam correlações significantes com os traços de personalidade associados à suas respectivas soft skills.
Portanto, elas podem ser utilizadas para identificar se um programador possui alguma dessas habilidades. É importante observar, que cada métrica traz essa informação através de um valor entre zero e 100, significando que quanto maior o valor da métrica, maior o nível de desenvolvimento da soft skill. Dessa maneira, é possível perceber as habilidades que são pontos fortes de um indivíduo, como também aquelas que precisam de melhoria. O nível da habilidade também pode ser útil para ordenar um conjunto de candidatos a vagas de emprego, por exemplo.

Apesar de não ter sido possível validar todas as métricas, a maioria delas pode ser aplicada para identificação automática de soft skills. Com isso, consideramos que nossos objetivos foram atingidos.
Vale ressaltar que nossos resultados foram encontrados através de um estudo de validação cujos passos foram propostos com base na análise das correlações existentes entre métricas de soft skills e traços de personalidades do FFM. Tais resultados estão focados na utilização do sistema de juiz online Huxley. Sendo possível utilizar-se dos conceitos sobre as soft skills, do mapeamento que propomos e dos passos do estudo de validação para replicar nossa estratégia em outros contextos. 

Em suma, esta dissertação traz diversas contribuições:

\begin{itemize}
 \item Ampliamos a compreensão sobre aspectos humanos relacionados ao desenvolvimento de software, chamando atenção para habilidades importantes, porém muitas vezes ignoradas ou não compreendidas; 

 \item Fizemos uma levantamento de importantes soft skills para o programador de software. Adicionalmente, apontamos o relacionamento das mesmas com teorias de personalidade e explicamos como elas se aplicam no contexto da profissão. Com isso, trazemos informações para o profissional, no reconhecimento das necessidades do ambiente de trabalho e na reflexão dos atributos que precisa desenvolver; para os educadores e aprendizes de programação, no processo de construção de suas habilidades; e para as empresas, no conhecimento das competências do profissional qualificado;
%e para as empresas podem utilizar as métricas como alternativa para identificar os profissionais que precisam contratar.

 \item Elaboramos uma estratégia automática para a identificação de soft skills, a partir do desenvolvimento de métricas. Também, descrevemos os passos de um estudo de validação para as mesmas, que originalmente as correlacionam com traços de personalidade.
Essa estratégia pode ser replicada em outros ambientes similares ao Huxley, onde as métricas propostas podem ser adaptadas, ou mesmo, onde novas métricas podem ser criadas e avaliadas seguindo os passos do estudo de validação;

 \item Desenvolvemos e validamos métricas para identificar automaticamente soft skills em programadores, oferecendo uma alternativa às empresas no processo de contratação de profissionais.
\end{itemize}

Como trabalhos futuros, propomos que as métricas validadas sejam implantadas no juiz online Huxley, constituindo uma nova funcionalidade de identificação de soft skills para o sistema. A página de perfil dos usuários pode exibir as principais habilidades de cada um, destacando-as através de ícones (como um selo ou conquista). Essa ideia pode incentivá-los a adquirir e desenvolver soft skills importantes.

Além disso, recomendamos a criação de uma página de busca de usuários por soft skills. Tal funcionalidade pode ser direcionada para empresas, de forma que o sistema os indicasse a empregos de programador, de acordo com suas habilidades mais desenvolvidas. No contexto do Huxley, isso é especialmente útil porque a maioria dos membros ainda são estudantes e não possuem experiência profissional. Portanto, eles não contam com recomendações de antigos empregos, mas poderiam utilizar as indicações do sistema como forma de demonstrar suas soft skills.

Aconselhamos também a implementação de funcionalidades como bate-papo, fóruns, grupos de estudos, comunidade de membros, etc., como forma de melhorar a interação entre os usuários do Huxley. Tais funcionalidades, além de incrementar o sistema, poderiam ser empregadas na melhoria da métrica Comunicação, por exemplo.

Tanto a referida métrica, quanto Trabalho independente, precisam ser revisadas. Propomos que pesquisas futuras sejam realizadas nesse sentido, sendo também possível trabalhar na melhoria das métricas que já demonstram aplicação.
%, a fim de atingir melhores níveis de correlação com os traços de personalidade associados.

Além disso, outras pesquisas podem ser desenvolvidas fora do contexto do Huxley, por exemplo, aplicando nossa estratégia em diferentes juízes online. As métricas que propomos podem ser extraídas ou adaptadas a outros sistemas, de acordo com as funcionalidades do mesmo. Em seguida, seria possível executar os passos do estudo de validação para testar as métricas em outros ambientes e assim estender a identificação de soft skills como um recurso automático para que empresas no processo de contratação de profissionais possam encontrar programadores capacitados.


