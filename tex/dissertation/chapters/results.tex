
%7%%7%%7%%7%%7%%7%%7%%7%%7%%7%%7%%7%%7%%7%%7%%7%%7%%7%%7
%                     Resultados                       %
%7%%7%%7%%7%%7%%7%%7%%7%%7%%7%%7%%7%%7%%7%%7%%7%%7%%7%%7

\chapter{Resultados}

\label{chap:results}

Neste Capítulo apresentamos os resultados do estudo de validação descrito no Capítulo \ref{chap:evaluation}. Inicialmente, na Seção \ref{sec:tipitipi} discutimos a correlação entre os itens do questionário TIPI para analisar a consistência das respostas dos participantes do estudo. Na Seção \ref{sec:tipiss}, mostramos as correlações encontradas entre as respostas do TIPI e os valores das métricas para identificação de soft skills. Em seguida, discutimos os resultados na Seção \ref{sec:discussao}.

\section{Correlações entre os itens do TIPI}
\label{sec:tipitipi}

De acordo com a proposta de Gosling et al. \cite{gosling:03}, os itens do TIPI denotam os cinco grandes fatores de personalidade (Big Five). Os itens estão organizados em pares, de forma que,
para cada fator existe um item positivo e um negativo.
Por exemplo, o item \textit{1. Extrovertido, entusiasmado} indica o traço positivo do fator Extroversão. Já o item \textit{6. Reservado, quieto} é o traço negativo desse fator, indicando Introversão.
Similarmente, a relação que existe entre os demais fatores e os itens é:
Amabilidade: \textit{2. Crítico, briguento} (negativo) e \textit{7. Compreensível, amável}; 
Conscienciosidade: \textit{3. Seguro, auto-disciplinado} e \textit{8. Desorganizado, descuidado} (negativo);
Neuroticismo: \textit{4. Ansioso, facilmente chateado} e \textit{9. Calmo, emocionalmente estável} (negativo);
Abertura à experiência: \textit{5. Aberto a novas experiências, complexo} e \textit{10. Convencional, não criativo} (negativo).

Quando um indivíduo responde o TIPI, ele auto-avalia sua personalidade pontuando os itens numa escala de 1 a 7, significando o valor 1 discordo fortemente, e 7, concordo fortemente.
Com isso, podemos entender que quanto maior for o número escolhido para os itens relativos ao traço positivo, menores serão os números respondidos em itens de traços negativos, e vice-versa. Assim, para que haja consistência nas respostas do TIPI, precisamos encontrar uma correlação negativa entre os pares de itens:

\begin{itemize}
\item 1. Extrovertido, entusiasmado vs. 6. Reservado, quieto;
\item 2. Crítico, briguento vs. 7. Compreensível, amável;
\item 3. Seguro, auto-disciplinado vs. 8. Desorganizado, descuidado;
\item 4. Ansioso, facilmente chateado vs. 9. Calmo, emocionalmente estável;
\item 5. Aberto a novas experiências, complexo vs. 10. Convencional, não criativo.
\end{itemize}

Depois de coletar as respostas do questionário dos 32 participantes (N = 32), examinamos as correlações dos 10 itens do TIPI entre si.
Fazemos isso a fim de analisar a consistência das respostas dos participantes. Observe a Tabela \ref{tab:tipitipi}.
Note que destacamos as correlações entre os itens do TIPI que são inversos. 

\begin{sidewaystable}[ph!]
\footnotesize
\caption{\small Correlação entre os itens do TIPI}
\addtolength{\tabcolsep}{1pt}
%\renewcommand{\arraystretch}{1.7} 
\centering

    \begin{tabular}{lcccccccccc}
    \toprule
          & \multicolumn{10}{c}{\textbf{Item}} \\
    \multicolumn{1}{c}{\textbf{Item}} & \textbf{1} & \textbf{6} & \textbf{2} & \textbf{7} & \textbf{3} & \textbf{8} & \textbf{4} & \textbf{9} & \textbf{5} & \textbf{10} \\
		\midrule
    \multicolumn{1}{l}{\textbf{1. Extrovertido, entusiasmado}} 						& -     &       &       &       &       &       &       &       &       &  \\
    \multicolumn{1}{l}{\textbf{6. Reservado, quieto}} 										& \textbf{-.29}	& -     &       &       &       &       &       &       &       &  \\
    \textbf{} & & & & & & & & & &  \\
    \multicolumn{1}{l}{\textbf{2. Crítico, briguento}} 										& -.25  & .11   & -     &       &       &       &       &       &       &  \\
    \multicolumn{1}{l}{\textbf{7. Compreensível, amável}} 								& .17   & -.39  & \textbf{-.48} & -     &       &       &       &       &       &  \\
    \textbf{} & & & & & & & & & &  \\
    \multicolumn{1}{l}{\textbf{3. Seguro, auto-disciplinado}} 						& .03   & -.14  & .22   & .04   & -     &       &       &       &       &  \\
    \multicolumn{1}{l}{\textbf{8. Desorganizado, descuidado}} 						& -.10  & -.01  & -.16  & .18   & \textbf{-.31} & -     &       &       &       &  \\
    \textbf{} & & & & & & & & & &  \\
    \multicolumn{1}{l}{\textbf{4. Ansioso, facilmente chateado}} 					& -.05  & -.03  & .44   & -.18  & .23   & -.05  & -     &       &       &  \\
    \multicolumn{1}{l}{\textbf{9. Calmo, emocionalmente estável}} 				& .08   & -.01  & -.31  & .18   & -.12  & .24   & \textbf{-.59} & -     &       &  \\
    \textbf{} & & & & & & & & & &  \\
    \multicolumn{1}{l}{\textbf{5. Aberto a novas experiências, complexo}} & .16   & -.27  & -.34  & .28   & .17   & .06   & -.07  & .24   & -     &  \\
    \multicolumn{1}{l}{\textbf{10. Convencional, não criativo}} 					& -.44  & .25   & .08   & -.26  & -.07  & .09   & -.10  & .04   & \textbf{.07} & - \\
    
		\bottomrule
		\textbf{} & & & & & & & & & &  \\
		\textit{Nota: N = 32} & & & & & & & & & &  \\
		
%\begin{tabular}{|lcccccccccc|}
%\hline
%							& \multicolumn{10}{c|}{\textbf{Item}}	 				\\ \hline
%\textbf{Item} & 1 & 2 & 3 & 4 & 5 & 6 & 7 & 8 & 9 & 10 		\\ \hline \hline 

%1. Extrovertido, entusiasmado						& $-$ 	 				&      &       					&       &  		 					& 		  &		   					&		   &  		 				 &  		\\ \hline 
%6. Reservado, quieto											& \textbf{-.29} & $-$  &       					&       &  		 					& 		  &		   					&		   &  		 				 &  		\\ \hline \hline 

%2. Crítico, briguento										& $-.25$ 				& .11  & $-$  					&       &  		 					& 		  &		   					&		   &  		 				 &  		\\ \hline 
%7. Compreensível, amável									& $.17$  				& -.39 & \textbf{-.48}  & $-$   &      					&  		  & 							&		   &  		 				 &  		\\ \hline \hline 
 
%3. Seguro, auto-disciplinado							& $.03$  				& -.14 &  .22  					& .04	  & $-$  					&  		  &  		 					& 		 &  		 				 &  		\\ \hline 
%8. Desorganizado, descuidado							& $-.10$ 				& -.01 & -.16  					& .18	  & \textbf{-.31} & $-$   & 		 					&  		 &  		 				 &  		\\ \hline \hline 
 
%4. Ansioso, facilmente chateado					& $-.05$ 				& -.03 &  .44  					& -.18  &  .23 					& -.05	& $-$  					&  		 &  		 				 &  		\\ \hline 
%9. Calmo, emocionalmente estável					&  $.08$ 				& -.01 & -.31  					&  .18	& -.12 					&  .24	& \textbf{-.59} & $-$  &  		 				 &  		\\ \hline \hline 

%5. Aberto a novas experiências, complexo	& $.16$  				& -.27 & -.34  					&  .28	&  .17 					&  .06	& -.07 					&  .24	& $-$  				 &  		\\ \hline 
%10. Convencional, não criativo						& $-.44$ 				&  .25 &  .08  					& -.26	& -.07 					&  .09	& -.10 					&  .04	& \textbf{.07} & $-$  \\ \hline 

\end{tabular}
\label{tab:tipitipi}
\end{sidewaystable}

Somente para o fator Abertura à experiência, a correlação entre os itens \textit{5. Aberto a novas experiências, complexo} e \textit{10. Convencional, não criativo} não é negativa e é muito próxima a 0. Isso indica que pode haver biases nas respostas dadas para os itens relativos a esse fator.
É importante considerar essa informação, pois esse problema pode afetar também nas correlações entre os mesmos itens e as métricas que aplicamos.
%Isso porque, as respostas desses itens no questionário não são totalmente confiáveis.
%portanto, isso pode gerar incoonsistência também quando analisamos sua correlação com os resultados das métricas que aplicamos.

Essa inconsistência pode ter ocorrido porque os participantes não entenderam os itens e/ou por limitações do próprio instrumento. Gosling et al. \cite{gosling:03}, apresentam uma tabela similar em seu artigo. A correlação entre os itens do fator Abertura à experiência é, relativamente, a menos significativa (-.28, N = 1799, p> .05).
Nas demais dimensões, as correlações são: Amabilidade: -.36; Concienciosidade: -.42; Extroversão: -.59; e Neuroticismo: -.61, (N = 1799, p> .05).
%É possível que a mudança de idioma tenha afetado a qualidade dos itens.

\section{Correlações entre as métricas e os itens do TIPI}
\label{sec:tipiss}

Examinamos as correlações entre as pontuações obtidas a partir das métricas para identificação das soft skills e os itens do TIPI para verificar como as métricas estão relacionadas com os fatores de personalidade. A relação que estamos procurando foi discutida no Capítulo \ref{chap:concepts} e sumarizada na Figura \ref{fig:mapeamento}.
Podemos observar as correlações obtidas na Tabela \ref{tab:tipiss}. Note que destacamos as correlações que esperadas. 

\begin{sidewaystable}[ph!]
\footnotesize
\caption{\small Correlações entre as métricas e os itens do TIPI}
\addtolength{\tabcolsep}{1pt}
%\renewcommand{\arraystretch}{1.7} 
\centering
\begin{tabular}{lccccc}

    \toprule		
		& \textbf{1. Extrovertido, }  & \textbf{2. Crítico, } & \textbf{3. Seguro, } 			 & \textbf{4. Ansioso, } 				& \textbf{5. Aberto a novas } \\
		& \textbf{entusiasmado} 			& \textbf{briguento} 		& \textbf{auto-disciplinado} & \textbf{facilmente chateado} & \textbf{experiências, complexo} \\
					
    \midrule
    \multicolumn{1}{c}{\textbf{Análise de problemas}} 		& .353  				 & -.250 & \textbf{.337} & .080  & .203 				 \\
    \multicolumn{1}{c}{\textbf{Resolução de problemas}}		& -.198  				 & .143  & \textbf{.416} & .169  & .291 				 \\
    \multicolumn{1}{c}{\textbf{Atenção a detalhes}} 			& .243  				 & -.035 & \textbf{.343} & .365  & .171 				 \\
    \multicolumn{1}{c}{\textbf{Aprendizagem rápida}} 			& -.050  				 & -.136 & \textbf{.393} & .040  & \textbf{.307} \\
    \multicolumn{1}{c}{\textbf{Persistência}} 						& -.080  				 & .087  & \textbf{.307} & -.010 & .178 				 \\
    \multicolumn{1}{c}{\textbf{Comunicação}} 							& \textbf{-.163} & .200  & .134  			   & -.076 & -.117 				 \\
    \multicolumn{1}{c}{\textbf{Trabalho independente}} 		& .030  				 & -.142 & -.064 				 & -.026 & -.026 				 \\ 
		
    \multicolumn{1}{c}{\textbf{}} & & & & &  \\
		
    \toprule
		& \textbf{6. Reservado, } & \textbf{7. Compreensível, } & \textbf{8. Desorganizado, } & \textbf{9. Calmo, } 						& \textbf{10. Convencional, } \\
		& \textbf{quieto} 				& \textbf{amável} 						& \textbf{descuidado}					& \textbf{emocionalmente estável} & \textbf{não criativo} \\ 
		
		\midrule
    \multicolumn{1}{c}{\textbf{Análise de problemas}} 		& -.257 				 & \textbf{.209}  & .038  & .114  & -.266 				 \\
    \multicolumn{1}{c}{\textbf{Resolução de problemas}} 	& -.178 				 & \textbf{-.042} & .138  & -.147 & .163 				   \\
    \multicolumn{1}{c}{\textbf{Atenção a detalhes}} 			& -.305 				 & .107 				  & -.011 & -.180 & \textbf{-.222} \\
    \multicolumn{1}{c}{\textbf{Aprendizagem rápida}} 			& -.165 				 & .018 				  & .308  & .058  & -.076  				 \\
    \multicolumn{1}{c}{\textbf{Persistência}} 					  & -.263 				 & .024 				  & .013  & -.077 & -.025 				 \\
    \multicolumn{1}{c}{\textbf{Comunicação}} 							& .030 				   & -.025 				  & .158  & -.009 & -.141 				 \\
    \multicolumn{1}{c}{\textbf{Trabalho independente}} 		& \textbf{.183}  & .073 				  & .229  & -.189 & .035 				   \\
    
		\bottomrule
		\multicolumn{1}{l}{\textbf{}} & & & & &  \\
		\multicolumn{1}{l}{\textit{Nota: N = 32}} & & & & &  \\
		
\end{tabular}
\label{tab:tipiss}
\end{sidewaystable}

\section{Discussão}
\label{sec:discussao}

Esperava-se que as métricas Análise de problemas e Resolução de problemas apresentassem correlações positivas com os itens \textit{3. Seguro, auto-disciplinado} e \textit{7. Compreensível, amável}. Encontramos resultados positivos para o fator de Conscienciosidade (.337 e .416). Quanto ao fator Amabilidade, a correlação positiva ocorre com a métrica Análise de problemas (.209). Apesar de não esperada, também encontramos correlação entre a métrica Análise de problemas e o item \textit{1. Extrovertido, entusiasmado }(.353).

As métricas Atenção a detalhes, Aprendizagem rápida e Persistência também mostram correlação positiva (.343, .393, e .307, respectivamente) com o item esperado, \textit{3. Seguro, auto-disciplinado}. Esse resultado indica que essas métricas identificam as soft skills relacionadas com fator Conscienciosidade.

Aprendizagem rápida e o item \textit{5. Aberto a novas experiências, complexo}, apresentam correlação positiva (.307), como esperado. Com isso, é possível observar que a métrica está associada com características de um indivíduo aberto a novas experiências e que aprecia aprender coisas novas. A métrica Trabalho independente também mostrou correlação positiva (.183) com o item esperado, \textit{6. Reservado, quieto}.

\subsection{Correlações inversas ou insignificantes}

Existem correlações esperadas que não fomos capazes de encontrar. A correlação entre a métrica Resolução de problemas e o item \textit{7. Compreensível, amável} é insignificante. É possível que não encontramos essa correlação porque a métrica apenas engloba traços de seu outro fator relacionado, Conscienciosidade, não apresentando características de Amabilidade. Sabemos que este fator é uma dimensão da personalidade sobre interação interpessoal, por outro lado, a métrica foi aplicada em um sistema virtual onde não há recursos de interação social. Isso também pode explicar porque a métrica Análise de problemas e o mesmo item têm uma correlação positiva, mas baixa.

A métrica Atenção a detalhes mostra uma correlação negativa com o item
%que representa o traço inverso de Abertura à experiência,
\textit{10. Convencional, não criativo}, no entanto, esperava-se uma correlação positiva.
%, que representa o fator Abertura a experiência de forma negativa.
Nesse caso, nos remetemos à inconsistência entre os itens do TIPI referentes a esse fator, como mencionado na Seção \ref{sec:tipitipi}.
%onde identificamos correlação insignificante.
Esse pode ser o motivo de termos identificado essa correlação de maneira inversa.
Com isso, não podemos afirmar se a métrica está ou não correlacionada com o traço de personalidade esperado.

Apesar de estar esperando correlação positiva, a métrica Comunicação apresenta correlação negativa com o item \textit{1. Extrovertido, entusiasmado}.
%Entendemos que, uma vez que a métrica apenas considera comentários em códigos-fonte para identificar a soft skill, então ela só está identificando habilidades de comunicação escrita. Assim, podemos também considerar traços de introversão. No entanto, a métrica tem correlação insignificante com a característica inversa de extroversão, \textit{6. Reservado, quieto}.
Essa métrica não apresenta resultados satisfatórios possivelmente porque utiliza poucos recursos para identificação da soft skill,
uma vez que apenas considera comentários em códigos-fonte para identificar habilidades de comunicação,
por falta de funcionalidades como chat ou fórum no Huxley.
%Assim, não podemos dizer por meio dessa métrica se um indivíduo tem habilidades de comunicação.

\subsection{Significância das correlações}

Para saber se os nossos resultados podem ser aplicados em uma população geral, precisamos aplicar inferência e analisar além dos nossos dados
%, uma vez que o tamanho da amostra de estudo foi de apenas 32.
Nesta subseção estamos mostrando os valores de \textit{p} para cada correlação esperada entre métricas e os itens do questionário TIPI.
Com \textit{p-value}, somos capazes de conhecer se os resultados encontrados não foram produzidos por dados aleatórios.
A Tabela \ref{tab:pvalue} resume as correlações encontradas e adiciona a informação \textit{p-value}.

\begin{sidewaystable}[ph!]
\footnotesize
\caption{\small Correlações entre as métricas e os itens do TIPI, p-value}
\addtolength{\tabcolsep}{1pt}
%\renewcommand{\arraystretch}{1.7} 
\centering
\begin{tabular}{lccc}

    \toprule
          & \textbf{1. Extrovertido, entusiasmado} & \textbf{3. Seguro, auto-disciplinado} & \textbf{5. Aberto a novas experiências, complexo} \\
    \midrule
    \multicolumn{1}{c}{\textbf{Análise de problemas}} 	& \textbf{0.353; p = .04} & \textbf{0.337; p = .06} & \textbf{} \\
    \multicolumn{1}{c}{\textbf{Resolução de problemas}} &													& \textbf{0.416; p = .02} & \textbf{} \\
    \multicolumn{1}{c}{\textbf{Atenção a detalhes}} 		&													& \textbf{0.343; p = .05} &  \\
    \multicolumn{1}{c}{\textbf{Aprendizagem rápida}} 		&													& \textbf{0.393; p = .02} & \textbf{0.307; p = .09} \\
    \multicolumn{1}{c}{\textbf{Persistência}} 					&													& \textbf{0.307; p = .08} &  \\
    \multicolumn{1}{c}{\textbf{Comunicação}} 						& -0.163; p = .37					&													&  \\
    \multicolumn{1}{c}{\textbf{Trabalho independente}} 	&													&													&  \\
		
          &  &  &  \\

		\toprule					
          & \textbf{6. Reservado, quieto} & \textbf{7. Compreensível, amável} & \textbf{10. Calmo, emocionalmente estável} \\
		\midrule			
    \multicolumn{1}{c}{\textbf{Análise de problemas}} 	&													& 0.209; p = .25 	&  \\
    \multicolumn{1}{c}{\textbf{Resolução de problemas}} &													& -0.042; p = .82 &  \\
    \multicolumn{1}{c}{\textbf{Atenção a detalhes}} 		&													&       					& -0.222; p = .22 \\
    \multicolumn{1}{c}{\textbf{Aprendizagem rápida}} 		&													&       					&  \\
    \multicolumn{1}{c}{\textbf{Persistência}} 					&													&       					&  \\
    \multicolumn{1}{c}{\textbf{Comunicação}} 						&													&       					&  \\
    \multicolumn{1}{c}{\textbf{Trabalho independente}}	& 0.183; p = .31 					&									&  \\
		
    \bottomrule
		\multicolumn{1}{l}{\textbf{}} & & & \\
		\multicolumn{1}{l}{\textit{Nota: N = 32}} & & & \\
    
		
\end{tabular}
\label{tab:pvalue}
\end{sidewaystable}

Note que a Tabela \ref{tab:pvalue} mostra os mesmos valores de correlação da Tabela \ref{tab:tipiss}. 
As células vazias e colunas ocultas são valores insignificantes e/ou que não precisamos considerar.
%por conta da falta de relacionamento entre as soft skills e os fatores de personalidade. ? Não é bem isso...
Apenas mostramos as correlações que esperamos, de acordo com as associações que existem entre as soft skills e os fatores de personalidade.
%Também observamos a correlação entre a métrica Análise de problemas e item \textit{1. Extrovertido, entusiasmado}.
Em destaque, apresentamos as correlações significativas, considerando p-valor inferior a 0,1 (N = 32; p < .1).
