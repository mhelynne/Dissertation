
%7%%7%%7%%7%%7%%7%%7%%7%%7%%7%%7%%7%%7%%7%%7%%7%%7%%7%%7
%                     Resultados                       %
%7%%7%%7%%7%%7%%7%%7%%7%%7%%7%%7%%7%%7%%7%%7%%7%%7%%7%%7

\chapter{Resultados}

\label{chap:results}

Neste capítulo apresentamos os resultados do estudo de validação descrito no Capítulo \ref{chap:evaluation}. Inicialmente, na Seção \ref{sec:tipitipi} discutimos a correlação entre os itens do questionário TIPI a fim de analisar a consistência das respostas dos participantes do estudo. Na Seção \ref{sec:tipiss}, mostramos as correlações encontradas entre as respostas do TIPI e os valores das métricas para identificação de soft skills. Em seguida, discutimos os resultados na Seção \ref{sec:discussao}. Além disso, na Seção \ref{sec:limitacoes}, tratamos das limitações e especificidades dos mesmos.

\section{Correlações entre os itens do TIPI}
\label{sec:tipitipi}

De acordo com a proposta de Gosling et al. (2003)\nocite{gosling:03}, os itens do TIPI denotam os cinco grandes fatores de personalidade (Big Five). Os itens estão organizados em pares, de forma que,
para cada fator existe um item positivo e um negativo.
Por exemplo, o item \textit{1. Extrovertido, entusiasmado} indica o traço positivo do fator Extroversão. Já o item \textit{6. Reservado, quieto} é o traço negativo, ou inverso, indicando Introversão.
Similarmente, a relação que existe entre os demais fatores e os itens é:
Amabilidade: \textit{2. Crítico, briguento} (negativo) e \textit{7. Compreensível, amável}; 
Conscienciosidade: \textit{3. Seguro, autodisciplinado} e \textit{8. Desorganizado, descuidado} (negativo);
Neuroticismo: \textit{4. Ansioso, facilmente chateado} e \textit{9. Calmo, emocionalmente estável} (negativo);
Abertura à experiência: \textit{5. Aberto a novas experiências, complexo} e \textit{10. Convencional, não criativo} (negativo).

Quando um indivíduo responde o TIPI, ele auto avalia sua personalidade pontuando os itens numa escala de 1 a 7, significando o valor 1 discordo fortemente, e 7, concordo fortemente.
Com isso, podemos entender que quanto maior for o valor escolhido para itens relativos ao traço positivo, menor será o valor respondido em itens dos respectivos traços negativos, e vice-versa.
Assim, para que haja consistência nas respostas do TIPI, precisamos encontrar uma correlação negativa entre os pares de itens:

\begin{itemize}
\item 1. Extrovertido, entusiasmado vs. 6. Reservado, quieto;
\item 2. Crítico, briguento vs. 7. Compreensível, amável;
\item 3. Seguro, autodisciplinado vs. 8. Desorganizado, descuidado;
\item 4. Ansioso, facilmente chateado vs. 9. Calmo, emocionalmente estável;
\item 5. Aberto a novas experiências, complexo vs. 10. Convencional, não criativo.
\end{itemize}

Depois de coletar as respostas do questionário dos 32 participantes (total da amostra, N = 32),
examinamos as correlações dos 10 itens do TIPI entre si.
Fazemos isso a fim de analisar a consistência das respostas.
Disponibilizamos os arquivos contendo os dados dessa análise para acesso em:
\href{https://github.com/mhelynne/ss-data}{\textit{github.com/mhelynne/ss-data}}.
Esclarecemos que esses arquivos contêm apenas os valores das respostas dos questionários TIPI, mas não identificam os participantes.
Na Tabela \ref{tab:tipitipi}, destacamos as correlações encontradas entre os itens do TIPI.
Note que destacamos as correlações entre aqueles que são inversos. 

%sidetable
\begin{sidewaystable}[ph!]
\footnotesize
\caption{\small Correlação entre os itens do TIPI}
%\addtolength{\tabcolsep}{1pt}
\renewcommand{\arraystretch}{1.4} 
\centering

    \begin{tabular}{lcccccccccc}
    \toprule
          & \multicolumn{10}{c}{\textbf{Item}} \\
    \multicolumn{1}{c}{\textbf{Item}} & \textbf{1} & \textbf{6} & \textbf{2} & \textbf{7} & \textbf{3} & \textbf{8} & \textbf{4} & \textbf{9} & \textbf{5} & \textbf{10} \\
		\midrule
    \multicolumn{1}{l}{\textbf{1. Extrovertido, entusiasmado}} 						& -     &       &       &       &       &       &       &       &       &  \\
    \multicolumn{1}{l}{\textbf{6. Reservado, quieto}} 										& \textbf{-.29}	& -     &       &       &       &       &       &       &       &  \\
    \textbf{} & & & & & & & & & &  \\
    \multicolumn{1}{l}{\textbf{2. Crítico, briguento}} 										& -.25  & .11   & -     &       &       &       &       &       &       &  \\
    \multicolumn{1}{l}{\textbf{7. Compreensível, amável}} 								& .17   & -.39  & \textbf{-.48} & -     &       &       &       &       &       &  \\
    \textbf{} & & & & & & & & & &  \\
    \multicolumn{1}{l}{\textbf{3. Seguro, autodisciplinado}} 						  & .03   & -.14  & .22   & .04   & -     &       &       &       &       &  \\
    \multicolumn{1}{l}{\textbf{8. Desorganizado, descuidado}} 						& -.10  & -.01  & -.16  & .18   & \textbf{-.31} & -     &       &       &       &  \\
    \textbf{} & & & & & & & & & &  \\
    \multicolumn{1}{l}{\textbf{4. Ansioso, facilmente chateado}} 					& -.05  & -.03  & .44   & -.18  & .23   & -.05  & -     &       &       &  \\
    \multicolumn{1}{l}{\textbf{9. Calmo, emocionalmente estável}} 				& .08   & -.01  & -.31  & .18   & -.12  & .24   & \textbf{-.59} & -     &       &  \\
    \textbf{} & & & & & & & & & &  \\
    \multicolumn{1}{l}{\textbf{5. Aberto a novas experiências, complexo}} & .16   & -.27  & -.34  & .28   & .17   & .06   & -.07  & .24   & -     &  \\
    \multicolumn{1}{l}{\textbf{10. Convencional, não criativo}} 					& -.44  & .25   & .08   & -.26  & -.07  & .09   & -.10  & .04   & \textbf{.07} & - \\
    
		\bottomrule
		\textbf{} & & & & & & & & & &  \\
		\textit{Nota: N = 32} & & & & & & & & & &  \\
		
%\begin{tabular}{|lcccccccccc|}
%\hline
%							& \multicolumn{10}{c|}{\textbf{Item}}	 				\\ \hline
%\textbf{Item} & 1 & 2 & 3 & 4 & 5 & 6 & 7 & 8 & 9 & 10 		\\ \hline \hline 

%1. Extrovertido, entusiasmado						& $-$ 	 				&      &       					&       &  		 					& 		  &		   					&		   &  		 				 &  		\\ \hline 
%6. Reservado, quieto											& \textbf{-.29} & $-$  &       					&       &  		 					& 		  &		   					&		   &  		 				 &  		\\ \hline \hline 

%2. Crítico, briguento										& $-.25$ 				& .11  & $-$  					&       &  		 					& 		  &		   					&		   &  		 				 &  		\\ \hline 
%7. Compreensível, amável									& $.17$  				& -.39 & \textbf{-.48}  & $-$   &      					&  		  & 							&		   &  		 				 &  		\\ \hline \hline 
 
%3. Seguro, autodisciplinado							& $.03$  				& -.14 &  .22  					& .04	  & $-$  					&  		  &  		 					& 		 &  		 				 &  		\\ \hline 
%8. Desorganizado, descuidado							& $-.10$ 				& -.01 & -.16  					& .18	  & \textbf{-.31} & $-$   & 		 					&  		 &  		 				 &  		\\ \hline \hline 
 
%4. Ansioso, facilmente chateado					& $-.05$ 				& -.03 &  .44  					& -.18  &  .23 					& -.05	& $-$  					&  		 &  		 				 &  		\\ \hline 
%9. Calmo, emocionalmente estável					&  $.08$ 				& -.01 & -.31  					&  .18	& -.12 					&  .24	& \textbf{-.59} & $-$  &  		 				 &  		\\ \hline \hline 

%5. Aberto a novas experiências, complexo	& $.16$  				& -.27 & -.34  					&  .28	&  .17 					&  .06	& -.07 					&  .24	& $-$  				 &  		\\ \hline 
%10. Convencional, não criativo						& $-.44$ 				&  .25 &  .08  					& -.26	& -.07 					&  .09	& -.10 					&  .04	& \textbf{.07} & $-$  \\ \hline 

\end{tabular}
\label{tab:tipitipi}
\end{sidewaystable}

Somente para o fator Abertura à experiência, a correlação entre os itens \textit{5. Aberto a novas experiências, complexo} e \textit{10. Convencional, não criativo} não é negativa e é muito próxima a 0. Isso indica que pode haver inconsistência nas respostas dadas para os itens relativos a esse fator.
%Isso porque, as respostas desses itens no questionário não são totalmente confiáveis.
%portanto, isso pode gerar incoonsistência também quando analisamos sua correlação com os resultados das métricas que aplicamos.

Essa inconsistência pode ter ocorrido porque os participantes não entenderam os itens e/ou por limitações do próprio instrumento. Gosling et al. (2003)\nocite{gosling:03} apresentam uma tabela similar em seu artigo. A correlação entre os itens do fator Abertura à experiência é, relativamente, a menos significativa (-.28, N = 1799, p> .05).
Para os demais fatores, as correlações são: Amabilidade: -.36; Conscienciosidade: -.42; Extroversão: -.59; e Neuroticismo: -.61, (N = 1799, p> .05).

É importante considerar essa inconsistência, pois ela indica que uma parte das respostas do questionário TIPI está comprometida, ou seja, os participantes podem não ter respondido corretamente os itens respectivos ao fator Abertura à experiência. Por consequência, isso pode interferir também nas correlações entre esses itens (\textit{5. Aberto a novas experiências, complexo} e \textit{10. Convencional, não criativo}) e as métricas relacionadas aos mesmos.
As correlações entre as métricas e os itens do TIPI são analisadas na seção a seguir.
%, de forma que, podemos nos deparar com correlações não significativas nessa dimensão, onde não será possível afirmar a relação entre a métrica e o fator de personalidade Abertura à experiência.
%É possível que a mudança de idioma tenha afetado a qualidade dos itens.

\section{Correlações entre as métricas e os itens do TIPI}
\label{sec:tipiss}

Examinamos as correlações entre as pontuações obtidas a partir das métricas para identificação das soft skills e os itens do TIPI a fim de verificar como as métricas estão relacionadas com os fatores de personalidade.
Os arquivos com os dados para essa análise estão disponíveis para acesso em:
\href{https://github.com/mhelynne/ss-data}{\textit{github.com/mhelynne/ss-data}}.
Vale ressaltar que os arquivos contêm apenas os valores resultantes das métricas de soft skills e as respostas dos questionários TIPI, mas não identificam os participantes.

A relação que procuramos foi discutida no Capítulo \ref{chap:concepts} e sumarizada na Figura \ref{fig:mapeamento}.
Podemos observar as correlações obtidas na Tabela \ref{tab:tipiss}. Note que destacamos as correlações esperadas. 

\begin{sidewaystable}[ph!]
\footnotesize
\caption{\small Correlações entre as métricas e os itens do TIPI}
%\addtolength{\tabcolsep}{1pt}
\renewcommand{\arraystretch}{1.4} 
\centering
\begin{tabular}{lccccc}

    \toprule		
		& \textbf{1. Extrovertido, }  & \textbf{2. Crítico, } & \textbf{3. Seguro, } 			 & \textbf{4. Ansioso, } 				& \textbf{5. Aberto a novas } \\
		& \textbf{entusiasmado} 			& \textbf{briguento} 		& \textbf{autodisciplinado} & \textbf{facilmente chateado} & \textbf{experiências, complexo} \\
					
    \midrule
    \multicolumn{1}{c}{\textbf{Análise de problemas}} 		& .353  				 & -.250 & \textbf{.337} & .080  & .203 				 \\
    %Todos os usuários como referência
		\multicolumn{1}{c}{\textbf{Resolução de problemas}}	  & -.198  				 & .143  & \textbf{.416} & .169  & .291 				 \\
		%Usuários da turma como referência
		%\multicolumn{1}{c}{\textbf{Resolução de problemas}}	& -.203  				 & .122  & \textbf{.398} & .187  & .278 				 \\		 
    \multicolumn{1}{c}{\textbf{Atenção a detalhes}} 			& .243  				 & -.035 & \textbf{.343} & .365  & .171 				 \\
    %Antes de arrumar o algoritmo do fastlearning
		%\multicolumn{1}{c}{\textbf{Aprendizagem rápida}} 		& -.050  				 & -.136 & \textbf{.393} & .040  & \textbf{.307} \\
		%Depois de arrumar o algoritmo do fastearning
		\multicolumn{1}{c}{\textbf{Aprendizagem rápida}} 			& .223   			   & -.031 & \textbf{.520} & .122  & \textbf{.394} \\
    \multicolumn{1}{c}{\textbf{Persistência}} 						& -.080  				 & .087  & \textbf{.307} & -.010 & .178 				 \\
    \multicolumn{1}{c}{\textbf{Comunicação}} 							& \textbf{-.163} & .200  & .134  			   & -.076 & -.117 				 \\
    \multicolumn{1}{c}{\textbf{Trabalho independente}} 		& .030  				 & -.142 & -.064 				 & -.026 & -.026 				 \\ 
		
    \multicolumn{1}{c}{\textbf{}} & & & & &  \\
		
    \toprule
		& \textbf{6. Reservado, } & \textbf{7. Compreensível, } & \textbf{8. Desorganizado, } & \textbf{9. Calmo, } 						& \textbf{10. Convencional, } \\
		& \textbf{quieto} 				& \textbf{amável} 						& \textbf{descuidado}					& \textbf{emocionalmente estável} & \textbf{não criativo} \\ 
		
		\midrule
    \multicolumn{1}{c}{\textbf{Análise de problemas}} 		& -.257 				 & \textbf{.209}  & .038  & .114  & -.266 				 \\
    %Todos os usuários como referência
		\multicolumn{1}{c}{\textbf{Resolução de problemas}} 	& -.178 				 & \textbf{-.042} & .138  & -.147 & .163 				   \\
		%Usuários da turma como referência
		%\multicolumn{1}{c}{\textbf{Resolução de problemas}} 	& -.153 				 & \textbf{-.026} & .173  & -.150 & .145 				   \\
    \multicolumn{1}{c}{\textbf{Atenção a detalhes}} 			& -.305 				 & .107 				  & -.011 & -.180 & \textbf{-.222} \\
    %Antes de arrumar o algoritmo do fastlearning
		%\multicolumn{1}{c}{\textbf{Aprendizagem rápida}} 		& -.165 				 & .018 				  & .308  & .058  & -.076  				 \\
		%Depois de arrumar o algoritmo do fastlearning
		\multicolumn{1}{c}{\textbf{Aprendizagem rápida}} 			& -.319 				 & -.112 				  & .096  & -.055 & -.113  				 \\
    \multicolumn{1}{c}{\textbf{Persistência}} 					  & -.263 				 & .024 				  & .013  & -.077 & -.025 				 \\
    \multicolumn{1}{c}{\textbf{Comunicação}} 							& .030 				   & -.025 				  & .158  & -.009 & -.141 				 \\
    \multicolumn{1}{c}{\textbf{Trabalho independente}} 		& \textbf{.183}  & .073 				  & .229  & -.189 & .035 				   \\
    
		\bottomrule
		\multicolumn{1}{l}{\textbf{}} & & & & &  \\
		\multicolumn{1}{l}{\textit{Nota: N = 32}} & & & & &  \\
		
\end{tabular}
\label{tab:tipiss}
\end{sidewaystable}

\section{Discussão}
\label{sec:discussao}

Esperava-se que as métricas Análise de problemas e Resolução de problemas apresentassem correlações positivas com os itens \textit{3. Seguro, autodisciplinado} e \textit{7. Compreensível, amável}. Encontramos resultados positivos para o fator de Conscienciosidade (.337 e .416). Quanto ao fator Amabilidade, a correlação positiva ocorre com a métrica Análise de problemas (.209).
%Apesar de não esperada, também encontramos correlação entre a métrica Análise de problemas e o item \textit{1. Extrovertido, entusiasmado} (.353).

As métricas Atenção a detalhes, Aprendizagem rápida e Persistência também mostram correlação positiva (.343, .520, e .307, respectivamente) com o item esperado, \textit{3. Seguro, autodisciplinado}. Esse resultado indica que essas métricas identificam as soft skills relacionadas com fator Conscienciosidade.

Aprendizagem rápida e o item \textit{5. Aberto a novas experiências, complexo}, apresentam correlação positiva (.394), como esperado. Com isso, é possível observar que a métrica está associada com características de um indivíduo aberto a novas experiências e que aprecia aprender coisas novas.
%A métrica Trabalho independente também mostrou correlação positiva (.183) com o item esperado, \textit{6. Reservado, quieto}.

\subsection{Correlações inversas ou insignificantes}

Existem correlações esperadas que não fomos capazes de encontrar. A correlação entre a métrica Resolução de problemas e o item \textit{7. Compreensível, amável} é insignificante. É possível que não encontramos essa correlação porque a métrica apenas engloba traços de seu outro fator relacionado, Conscienciosidade, não apresentando características de Amabilidade. Sabemos que este fator é uma dimensão da personalidade sobre interação interpessoal, por outro lado, a métrica foi aplicada em um sistema virtual onde não há recursos de interação social. Isso também pode explicar porque a métrica Análise de problemas e o mesmo item têm uma correlação positiva, mas baixa.

A métrica Atenção a detalhes mostra uma correlação negativa com o item
%que representa o traço inverso de Abertura à experiência,
\textit{10. Convencional, não criativo}, no entanto, esperava-se uma correlação positiva.
%, que representa o fator Abertura a experiência de forma negativa.
Nesse caso, nos remetemos à inconsistência entre os itens do TIPI referentes a esse fator, como mencionado na Seção \ref{sec:tipitipi}.
%onde identificamos correlação insignificante.
Esse pode ser o motivo de termos identificado essa correlação de maneira inversa.
Com isso, não podemos afirmar se a métrica está ou não correlacionada com o traço de personalidade esperado.

Apesar de esperarmos correlação positiva, a métrica Comunicação apresenta correlação negativa com o item \textit{1. Extrovertido, entusiasmado}.
%Entendemos que, uma vez que a métrica apenas considera comentários em códigos-fonte para identificar a soft skill, então ela só está identificando habilidades de comunicação escrita. Assim, podemos também considerar traços de introversão. No entanto, a métrica tem correlação insignificante com a característica inversa de extroversão, \textit{6. Reservado, quieto}.
Essa métrica não apresenta resultados satisfatórios possivelmente porque utiliza poucos recursos para identificação da soft skill,
uma vez que apenas considera comentários em códigos-fonte para identificar habilidades de comunicação,
por falta de funcionalidades como bate-papo ou fórum no Huxley.
%Assim, não podemos dizer por meio dessa métrica se um indivíduo tem habilidades de comunicação.

A métrica Trabalho independente mostrou correlação positiva (.183) com o item \textit{6. Reservado, quieto}. No entanto, esse valor é baixo, mostrando que a métrica não apresenta uma correlação significativa com o traço de personalidade esperado, por isso, consideramos que essa métrica precisa de revisão para melhor identificar a respectiva soft skill.  

\subsection{Significância das correlações}

%Para saber se os nossos resultados podem ser aplicados em uma população geral, precisamos aplicar inferência e analisar além dos nossos dados.
%, uma vez que o tamanho da amostra de estudo foi de apenas 32.
Nesta seção mostramos os valores de \textit{p} para cada correlação esperada entre as métricas e os itens do questionário TIPI.
Com \textit{p-value}, somos capazes de reconhecer se os resultados encontrados não foram produzidos por dados aleatórios.
A Tabela \ref{tab:pvalue} resume as correlações encontradas e adiciona a informação \textit{p-value}.

\begin{sidewaystable}[ph!]
\footnotesize
\caption{\small Correlações entre as métricas e os itens do TIPI, \textsl{p-value}}
%\addtolength{\tabcolsep}{1pt}
\renewcommand{\arraystretch}{1.4} 
\centering
\begin{tabular}{lccc}

    \toprule
          & \textbf{1. Extrovertido, entusiasmado} & \textbf{3. Seguro, autodisciplinado} & \textbf{5. Aberto a novas experiências, complexo} \\
    \midrule
    \multicolumn{1}{c}{\textbf{Análise de problemas}} 	& 						  						& \textbf{.337; \textsl{p} = .06} & \textbf{} \\
    \multicolumn{1}{c}{\textbf{Resolução de problemas}} &													  & \textbf{.416; \textsl{p} = .02} & \textbf{} \\
    \multicolumn{1}{c}{\textbf{Atenção a detalhes}} 		&														& \textbf{.343; \textsl{p} = .05} & \\
    \multicolumn{1}{c}{\textbf{Aprendizagem rápida}} 		&														& \textbf{.520; \textsl{p} = .002} & \textbf{.394; \textsl{p} = .02} \\
    \multicolumn{1}{c}{\textbf{Persistência}} 					&														& \textbf{.307; \textsl{p} = .09} & \\
    \multicolumn{1}{c}{\textbf{Comunicação}} 						& -.163; \textsl{p} = .37		&																  & \\
    \multicolumn{1}{c}{\textbf{Trabalho independente}} 	&														&															    & \\
		
          &  &  &  \\

		\toprule					
          & \textbf{6. Reservado, quieto} & \textbf{7. Compreensível, amável} & \textbf{10. Calmo, emocionalmente estável} \\
		\midrule			
    \multicolumn{1}{c}{\textbf{Análise de problemas}} 	&													& .209;  \textsl{p} = .25  &  \\
    \multicolumn{1}{c}{\textbf{Resolução de problemas}} &													& -.042; \textsl{p} = .82  &  \\
    \multicolumn{1}{c}{\textbf{Atenção a detalhes}} 		&													&       									 & -.222; \textsl{p} = .22 \\
    \multicolumn{1}{c}{\textbf{Aprendizagem rápida}} 		&													&       									 &  \\
    \multicolumn{1}{c}{\textbf{Persistência}} 					&													&       									 &  \\
    \multicolumn{1}{c}{\textbf{Comunicação}} 						&													&       									 &  \\
    \multicolumn{1}{c}{\textbf{Trabalho independente}}	& .183; \textsl{p} = .31  &       									 &  \\
		
    \bottomrule
		\multicolumn{1}{l}{\textbf{}} & & & \\
		\multicolumn{1}{l}{\textit{Nota: N = 32}} & & & \\
    
		
\end{tabular}
\label{tab:pvalue}
\end{sidewaystable}

Note que a Tabela \ref{tab:pvalue} mostra os mesmos valores de correlação da Tabela \ref{tab:tipiss}. 
As células vazias e colunas ocultas são valores insignificantes e/ou que não precisamos considerar.
%por conta da falta de relacionamento entre as soft skills e os fatores de personalidade. ? Não é bem isso...
Apenas mostramos as correlações que esperamos, de acordo com as associações que existem entre as soft skills e os fatores de personalidade.
%Também observamos a correlação entre a métrica Análise de problemas e item \textit{1. Extrovertido, entusiasmado}.
Em destaque, apresentamos as correlações significativas, considerando \textit{p-value} inferior a 0,1 (N = 32; p < .1).

Diante destes resultados, observamos que as correlações entre as métricas Análise de problemas e Resolução de problemas e o item relativo ao fator Amabilidade não são significativas. Justificamos que isso ocorre devido a aplicação das mesmas em um sistema virtual, onde não há recursos de interação social que proporcione a medição de características do fator Amabilidade. Já a métrica Atenção a detalhes e sua correlação com o fator Abertura à experiência pode ter sido afetado pela inconsistência das respostas nessa dimensão do questionário TIPI. O que também pode ser observado junto a menor significância (maior \textit{p-value}) entre Aprendizagem rápida e o outro item do mesmo fator.

Por outro lado, encontramos que as métricas Análise de problemas, Resolução de problemas, Atenção a detalhes, Aprendizagem rápida e Persistência apresentam correlações significativas com um traço de personalidade esperado, ou seja, com o fator Conscienciosidade de forma positiva.
Observamos ainda correlação entre Aprendizagem rápida e Abertura à experiência.
Com isso, indicamos que essas métricas podem ser utilizadas para identificação das respectivas soft skills.
Já Comunicação e Trabalho independente não apresentam nenhuma correlação esperada de forma significante, portanto, recomendamos que essas métricas não são indicadas para identificação, necessitando serem revisadas para esse fim.

\section{Limitações}
\label{sec:limitacoes}

Diante deste estudo de validação, precisamos discutir o contexto de nossos resultados, suas limitações, bem como formas de contorná-las. Tratamos sobre esses assuntos nesta seção.

Inicialmente, ressaltamos que o estudo foi aplicado em um juiz online específico, ou seja, o Huxley, utilizando um subconjunto de seus usuários. Portanto, apresentamos os resultados de forma específica a esse contexto. Assim, esclarecemos que não utilizamos técnicas de generalização dos resultados para uma população geral. %os resultados não generalizam a aplicação das métricas para outros juízes online e

No entanto, a pesquisa que apresenta os conceitos das soft skills, o mapeamento com traços de personalidade e os passos do estudo de validação podem ser aplicados em diferentes contextos.
%, ou seja, em outros juízes online ou com diferentes métricas. 
%É possível utilizar nossa estratégia para identificação automática de soft skills a partir do desenvolvimento de métricas que, similarmente, podem ser aplicadas em outros contextos.
Caso seja possível extrair os dados que completem os valores necessários para o cálculo de cada métrica, pode-se aplicá-las em outros juízes online, ou ainda em diferentes tipos de sistemas ou ferramentas que permitam a observação de atividades de programação, como por exemplo, produções de um time de desenvolvimento em repositório, IDE instrumentado, entre outros.

Diante da aplicação das métricas em outro juiz online, ou em outro tipo de sistema, os responsáveis por tal aplicação devem inicialmente considerar o que cada métrica requer para ser calculada, determinando se seu sistema possui os recursos e/ou funcionalidades para extração das mesmas. Após essa verificação, podendo uma métrica ser implementada, em seguida faz-se necessário sua validação diante do novo contexto. Para isso, esta pesquisa traz informações a respeito das soft skills em seus conceitos, bem como o mapeamento das mesmas com os traços de personalidade e o passo-a-passo de como conduzir um estudo de validação e comparar as correlações do resultado a partir do mapeamento.

Com isso, consideramos que apesar de nosso estudo ser específico para um grupo de usuários do Huxley, o mesmo pode ser adaptado para diferentes contextos, ressaltando o cuidado de ser necessário a condução do estudo de validação para a análise das métricas em outras ocasiões, por exemplo, com outro grupo de usuários do Huxley, em outro juiz online, ou até mesmo, com outros tipos de sistemas. 

Ainda sobre sua aplicação específica ao contexto do Huxley, sabemos que os usuários que fazem parte desse sistema são ainda estudantes de programação e, portanto, estão em formação profissional. A medição das soft skills obtida através do Huxley ocorreu no primeiro ano do curso, no entanto, devemos considerar que é possível que durante o decorrer dos estudos, os estudantes adquiram ou melhorem suas soft skills. Tais progressos não seriam expressos pela soft skill medida inicialmente.

Isso representa uma necessidade por atualizar os valores identificados para as soft skills. Porém, em fases posteriores do curso, o Huxley não é comumente utilizado como juiz online. Para contornar essa limitação, mencionamos, mais uma vez, a possibilidade de aplicar nossa estratégia em outros juízes online ou em diferentes ambientes respectivos a outros períodos da carreira do programador. Ratificando que as métricas que propomos podem ser adaptadas a outros contextos e buscando a validação das mesmas através da execução do estudo de validação que apresentamos.

Apesar de estar limitado a fases iniciais do curso e voltado a aprendizagem de programação, a utilização do juiz online Huxley para medição das soft skills de forma automática continua relevante. Frisamos que a medição nesse período do curso pode ser útil para contratações em estágios, quando as empresas encontram mais dificuldades para obter informações sobre seus candidatos, pois são em geral estudantes sem experiência profissional, restando apenas a experiência que os candidatos adquiriram em sua formação educacional, o que pode ser representado pelas medições das soft skills no Huxley.

Destacamos ainda a importância do conhecimento a respeito de soft skills por parte de um indivíduo desde o início de sua carreira, nesse caso ainda enquanto estudante programador, permitindo que o mesmo busque a aplicação de suas habilidades fortes e o desenvolvimento de habilidades que precisam de melhorias. 

