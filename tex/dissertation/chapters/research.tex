
%3%%3%%3%%3%%3%%3%%3%%3%%3%%3%%3%%3%%3%%3%%3%%3%%3%%3%%3
%               Levantamento de soft skills            %
%3%%3%%3%%3%%3%%3%%3%%3%%3%%3%%3%%3%%3%%3%%3%%3%%3%%3%%3

\chapter{Levantamento de soft skills}

\label{chap:research}

Este capítulo é um levantamento bibliográfico baseado em diversos estudos que tratam da demanda e importância das soft skills no contexto dos profissionais de Tecnologia da Informação (TI) e Sistemas de Informação (SI). O objetivo é reconhecer quais são as soft skills necessárias dessa área, focando no papel do programador de software.

Os trabalhos que referimos neste capítulo discutem o tema das soft skills por meio de quatro abordagens diferentes. 
Inicialmente, na seção \ref{sec:anuncios}, tratamos de artigos que utilizam anúncios de emprego para identificar a demanda por habilidades profissionais, incluindo soft skills. Na seção \ref{sec:entrevista}, citamos uma pesquisa que envolve entrevistas com membros da faculdade e da indústria para descobrir quais soft skills são consideradas diferenciais de sucesso para um programador. Apresentamos, na seção \ref{sec:curriculos}, o que alguns estudos e currículos de cursos de TI e SI ressaltam como soft skills necessárias para a formação do estudante de graduação da área. Por fim, na seção \ref{sec:mapeamento}, fazemos referências a trabalhos que apontam soft skills e as relacionam com teorias da personalidade.

O resultado desse levantamento bibliográfico é a definição das soft skills com que trabalhamos durante o restante desta pesquisa. Portanto, ao final deste capítulo, na seção \ref{sec:lista-ss}, propomos uma lista das soft skills requiridas para o profissional de programação.

\section{Anúncios de emprego}
\label{sec:anuncios}

Muitos estudos sobre a importância e demanda das soft skills têm sido realizados, por exemplo, Ahmed et al. \cite{ahmed:12} fazem uma análise de 500 anúncios para cargos de TI. Eles se concentram nas soft skills mencionadas nos anúncios para determinar quais estão em alta demanda no setor de desenvolvimento de software. Os anúncios foram coletados a partir de portais de recrutamento online da América do Norte, Europa, Ásia e Austrália. Eles incluem trabalhos nos cargos de analista de sistemas, designer, programador de computador e testador de software.

No que diz respeito ao papel do programador de computador, os autores apontam que as habilidades de comunicação estão em alta demanda, solicitada por 90\% dos anúncios. Há demanda moderada para habilidades de análise e resolução de problemas e para trabalho independente (52\% e 34\%). Os autores também destacam que o programador deve ser atento a detalhes, aprender rápido e ser inovador.

Para Ahmed et al. \cite{ahmed:12}, comunicação é a capacidade de transmitir informação de uma forma que ela seja bem recebida e compreendida. A habilidade de análise e resolução de problemas significa compreender, articular e resolver problemas complexos, tomando decisões sensatas com base nos dados disponíveis. Trabalho independente é poder realizar tarefas com uma supervisão mínima. Atenção a detalhes precisa ser aplicada para compreender o projeto do software e traduzi-lo em código executável. Aprender rápido é ser capaz de absorver novos conceitos e tecnologias em um espaço de tempo relativamente curto. E ser inovador é encontrar soluções novas e criativas.

Outro estudo que utiliza anúncios de empregos, por Lee \cite{lee:05}, coleta 902 anúncios a partir de 230 grandes organizações que fizeram parte da Fortune 500 durante os anos de 2001 e 2003. Lee procura identificar as hard skills e soft skills que os analistas de sistemas precisam. Seus resultados mostram que as habilidades referentes ao papel do desenvolvedor de software são as mais procuradas nesse setor.

Sobre soft skills, Lee aponta que comunicação (71,6\%) e habilidades interpessoais (66\%) demonstram alta demanda.
A capacidade de trabalhar de forma independente e auto-motivação (29,4\%) apresentam demanda média.
Além disso, mais de um quarto dos anúncios de emprego buscam a habilidade de resolução de problemas.

Kennan et al. \cite{kennan:09} também fazem uma análise de 400 anúncios de emprego, dessa vez, o foco está em cargos iniciais da carreira na área de SI. Os resultados constatam que nessa fase existe uma alta demanda para a área de desenvolvimento de sistemas (78\%), onde encontramos por exemplo o cargo de programador. Esse estudo ressalta a importância de entender as necessidades e expectativas dos empregadores e suas implicações diretas na formação profissional de graduandos.

O artigo aponta muitas habilidades técnicas e tópicos de conhecimento em computação. Os anúncios de emprego solicitam habilidades de comunicação, que aparecem em 73,8\% deles, e diversas características pessoais, contabilizando demanda em 68,3\%. Dentre as soft skills mencionadas estão adaptação, atenção a detalhes, criatividade, aprendizagem rápida, desejo por aprender, organização, resolução de problemas, cooperação, auto-motivação, trabalho independente, etc. 

Embora os estudos acima nos permitam uma visão geral das soft skills mais importantes para os programadores, há algumas limitações com relação a procurar soft skills em anúncios de emprego. Segundo Litecky \cite{litecky:04}, anúncios de empregos são focados no processo de recrutamento, ou filtragem de candidatos. Como mostrado anteriormente na Figura \ref{fig:modelocontratacao}, essa é a fase onde se procura por habilidades técnicas.

Por isso, nem sempre é comum que os empregadores solicitem soft skills em anúncios de emprego.
Normalmente, elas são investigadas em uma fase posterior do processo de contratação, na etapa de escolha, o que pode ocorrer com base em entrevistas ou recomendações, por exemplo. Isso significa que as soft skills aparecem algumas vezes em anúncios de emprego, mas a demanda real deve ser ainda maior e pode não ser bem representada por eles.

\section{Entrevista com faculdade e indústria}
\label{sec:entrevista}

Através de uma abordagem diferente, Sterling e Brinthaupt \cite{sterling:03} entrevistam membros do corpo docente da faculdade a fim de obter uma lista de habilidades que contribuem para o sucesso do programador que trabalha individualmente e do programador que trabalha em grupo. Em seguida, membros do corpo docente e da indústria avaliam a importância dessas habilidades, considerando cada caso. Seus resultados apontam algumas soft skills.

De acordo com eles, um programador de sucesso individual tem fortes habilidades de resolução de problemas técnicos, atenção a detalhes, persistência e auto-disciplina para trabalhar sozinho. Um programador de equipe de sucesso é hábil em relações interpessoais, cooperação e comunicação, mostra auto-confiança e consciência.

Esse resultado lista um conjunto de soft skills consideradas relevantes para o sucesso de programadores em diferentes tipos de contextos e sob diferentes pontos de vista. No entanto, o estudo de Sterling e Brinthaupt, e também aqueles que foram referidos anteriormente, não apresentam um conceito exploratório sobre cada soft skill. São dadas apenas breves descrições que não fornecem informação suficiente que para ser aplicada na identificação dessas habilidades em indivíduos.
%Portanto, nosso objetivo é passar por teorias psicológicas para explicar essas características humanas.

\section{Currículo e formação}
\label{sec:curriculos}

Snoke e Underwood \cite{snoke:01} analisam modelos de currículos para programas de graduação em Sistemas de Informação e áreas afins,
como  Guidelines for Undergraduate Degree Programs in Information Systems (IS’97) \cite{is97}
e     Information Systems-Centric Curriculum (ISCC'99) \cite{iscc99},
e um documento que descreve as necessidades da indústria de tecnologia na Austrália,
Australian Computer Society Core Body of Knowledge \cite{acs},
com o objetivo de identificar os principais atributos genéricos que devem ser considerados na formação dos estudantes como futuros profissionais.

Os autores utilizam o termo atributo genéricos para descrever um conjunto de capacidades de um indivíduo. Muitos atributos genéricos apontados pelos autores estão relacionados a habilidades não-técnicas. Assim, identificamos as soft skills resolução de problemas, trabalho independente, apredizagem rápida e comunicação. Apesar dos autores não empregarem esses termos específicos, podemos extraí-los a partir das declarações dos atributos genéricos que são apresentados:

\begin{itemize}
	\item Definir problemas de uma maneira sistemática, analisar, sintetizar e avaliar as várias soluções, considerando a qualidade delas são atributos para a habilidade de resolução de problemas.
	\item Confiança em aprender e desempenhar tarefas de forma independente e auto-motivação são atributos para a soft skill trabalho independente.
	\item	A aprendizagem rápida é uma habilidade de quem possui o atributo de curiosidade sobre tecnologias e capacidade de desenvolver o intelecto em um aprendizado contínuo.
	\item	A soft skill comunicação pode ser considerada como a habilidade de saber se comunicar escrita e oralmente, trabalhando como parte de um time de forma produtiva e cooperativa.
\end{itemize}

Os resultados de Snoke e Underwood \cite{snoke:01} foram utilizados em estudos posteriores \cite{snoke:02} e \cite{snoke:03} para comparar currículos de cursos de universidades australianas com os referidos modelos de currículo e com as necessidades da indústria. 
%Esses estudos ressaltam a necessidade dos currículos de cursos de SI e áreas afins serem explícitos em identificar as soft skills dos estudantes e procurar desenvolvê-las durante o curso.
Esses estudos ressaltam que o objetivo da educação superior é equipar os graduandos com os atributos exigidos pelo mercado de trabalho, incluindo aqueles que não são técnicos. Portanto, os educadores precisam ser capazes de identificar as competências de seus alunos e a universidade deve prover uma educação profissional que esteja aliada com as necessidades do mercado.

Em uma versão mais recente do modelo de currículo Guidelines for Undergraduate Degree Programs in Information Systems \cite{is10}, de 2010, o documento aborda principalmente habilidades técnicas a serem desenvolvidas nos estudantes e os perfis de cursos na área de SI. No entanto, é destacada a importância de que todos os cursos precisam desenvolver uma estrutura de pensamento para resolução de problemas, além de fortalecer habilidades de comunicação oral e escrita. Os estudantes também precisam entender que o curso e a profissão exigem persistência, curiosidade, criatividade, etc. O documento ISCC'99 também menciona essas soft skills e suporta esses requisitos.

\section{Mapeamento com teorias da personalidade}
\label{sec:mapeamento}

Em seu artigo, Rehman et al. \cite{rehman:12} atribuem algumas soft skills a profissionais do setor de desenvolvimento de software, abordando os papéis de analista de software, designer, programador, testador e manutenção. De uma forma geral, eles tratam das habilidades de comunicação, habilidades interpessoais, ouvinte ativo, aberto e adaptável a mudanças, inovador, habilidades de organização, aprendizagem rápida, trabalho em equipe, capacidade de trabalhar de forma independente, fortes habilidades analíticas e de resolução de problemas e atenção a detalhes. Especificamente, para os programadores eles indicam as três últimas soft skills como relevantes.

Os autores associaram essas soft skills com o cinco grandes traços da personalidade (Big Five). De acordo com eles, a capacidade de trabalhar de forma independente está relacionada com Extroversão, porém de uma forma inversa. Fortes habilidades analíticas e de resolução de problemas, com Amabilidade. Atenção a detalhes está relacionada com Abertura à experiência, também de forma inversa. Esse mapeamento foi feito com base em um estudo anterior, de Capretz e Ahmed \cite{capretz:10}, que também indica as mesmas soft skills.

Dessa forma, a respeito da personalidade do programador, as considerações de Rehman et al. \cite{rehman:12} e recomendações de outros estudos \cite{sodiya:07} \cite{martinez:11}, indicam que esse profissional possui um nível de Extroversão baixo, ou seja, é tipicamente introvertido, se sentindo a vontade em trabalhar sozinho e desenvolver suas tarefas sem necessidade de constante supervisão. Além disso, também possui um alto fator de Amabilidade, sendo alguém que é confiável e cooperativo. O fator Consienciosidade é indicado de forma positiva e Neuroticismo de forma negativa.

A respeito do fator Abertura à experiência, Rehman et al. \cite{rehman:12} e Sodiya et al. \cite{sodiya:07}, recomendam que seu nível seja baixo, por conta da necessidade de prestar atenção a detalhes de implementação e construção do software, que é uma característica oposta a esse fator. Em contrapartida, Martínez et al. \cite{martinez:11} recomenda que o nível de Abertura à experiência seja alto, permitindo características como inovação e criatividade.

Interpretamos que ambas as recomendações são bem-vindas no profissional de programação e com isso podemos mencionar que não existe o programador perfeito, aquele que seja equipado com todas as habilidades possíveis, pois muitas delas podem ir de encontro à personalidade. Ou seja, cada indivíduo terá os próprios pontos fortes, permitindo a composição de uma equipe de desenvolvimento de software variada, onde cada um poderá se destacar com suas principais habilidades. Vale ressaltar ainda que conhecer o máximo sobre as soft skills e conhecer a si próprio é importante para perceber as situações em que será preciso a melhoria de alguma habilidade e o esforço por desenvolvê-la.

Esses estudos foram úteis para entender como a personalidade está associada com as soft skills do programador de software, no entanto, eles só apontam três habilidades importantes para esse profissional. Temos a intenção de ampliar essa lista para envolver outras soft skills.

\section{Lista de soft skills}
\label{sec:lista-ss}

Com base nas referências encontradas na literatura sobre a demanda e a importância das soft skills, resumimos na Tabela \ref{tab:softskills}, as habilidades que escolhemos para trabalhar nesta pesquisa. Como é possível observar, elas foram selecionadas porque apresentaram-se através dos diversos estudos que tratamos durante o levantamento bibliográfico.

\begin{table*}[h]
\footnotesize
\caption{\small Soft skills apontadas na literatura para profissionais da área de TI} 
\addtolength{\tabcolsep}{-3.5pt}
\centering

		\begin{tabular}{|p{7cm}|c|c|c|c|c|c|c|c|c|}\hline
		
        \multicolumn{1}{|r|}{\bf Referências} & 1 & 2 & 3 & 4 & 5 & 6 & 7 & 8 & 9 \\\hline
				
{\bf Comunicação} & \ding{51} & \ding{51} & \ding{51} & \ding{51} & \ding{51} & \ding{51} & \ding{51} & \ding{51} & \ding{51}
\\\hline
{\bf Trabalho independente} & \ding{51} & \ding{51} & \ding{51} & \ding{51} & \ding{51} & \ding{51} & \ding{51} & & 
\\\hline
{\bf Atenção a detalhes}  & \ding{51} & \ding{51} &  & \ding{51} & \ding{51} & \ding{51} & & &
\\\hline
{\bf Análise e resolução de problemas} & \ding{51} & \ding{51} & \ding{51} & \ding{51} & \ding{51} & \ding{51} & \ding{51}  & \ding{51} & \ding{51}
\\\hline
{\bf Persistência} &  & \ding{51} &  &  &  &  &  & \ding{51} & \ding{51} 
\\\hline
{\bf Aprendizagem rápida} & \ding{51} &  &  & \ding{51} & \ding{51} & \ding{51} & \ding{51} & &
\\\hline

     \end{tabular}
		\label{tab:softskills}
		\fonte{1 - \cite{ahmed:12}; 2 - \cite{sterling:03}; 3 - \cite{lee:05}; 4 - \cite{kennan:09}; 
			5 - \cite{rehman:12}; 6 - \cite{capretz:10}; 7 - \cite{snoke:01}; 8 - \cite{iscc99}; 9 - \cite{is10}}
\end{table*}

Dessa maneira, damos destaque para uma lista de seis soft skills que consideramos serem relevantes para a qualificação profissional dos programadores de software: Análise e resolução de problemas, Atenção a detalhes, Aprendizagem rápida, Persistência, Comunicação, Trabalho independente.

Essa lista de soft skills envolve diferenciais para integração, permanência e crescimento do programador como profissional. No entanto, é preciso ressaltar que ela não tem objetivo de ser exaustiva, e não o é. Podemos listar soft skills importantes para qualquer profissional, mas é sempre possível considerar mais alguma. Portanto, para fins desta pesquisa, definimos a lista proposta neste capítulo como escopo.

%Desse ponto em diante, iremos explicar porque cada uma dessas soft skills é importante e como se aplicam no ambiente de trabalho do programador. Além disso, propomos um conceito sobre cada uma delas, através de um mapeamento com o Modelo dos Cinco Fatores. 