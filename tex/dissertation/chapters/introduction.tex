
%1%%1%%1%%1%%1%%1%%1%%1%%1%%1%%1%%1%%1%%1%%1%%1%%1%%1%%1
%                     Introdução                       %
%1%%1%%1%%1%%1%%1%%1%%1%%1%%1%%1%%1%%1%%1%%1%%1%%1%%1%%1

\begin{center}

\end{center}
\chapter{Introdução}

\label{chap:introduction}

Dados os avanços da tecnologia, a popularização do uso de computadores e aplicação da informática em diversas áreas como ciência, educação, saúde e etc., o mercado de trabalho do setor de Tecnologia da Informação (TI) apresenta alta demanda por profissionais. Segundo um estudo encomendado pela Cisco na América Latina e realizado pela empresa de consultoria IDC \cite{cisco:13}, o Brasil já é o 4º maior centro de TI do mundo, ficando atrás apenas dos Estados Unidos, China e Japão. TI é o segmento do mercado brasileiro que mais cresce. 

% www.cisco.com/assets/csr/pdf/IDC_Skills_Gap_-_LatAm.pdf

O estudo aponta que a demanda por profissionais de Tecnologia da Informação e Comunicação (TIC) no Brasil excederá a oferta em 32\% no ano de 2015. Em 2014, aproximadamente 100 mil vagas de trabalho foram abertas e esse número chegará a 120 mil em 2015. Essa demanda emergente implica em uma necessidade de qualificação dos profissionais da área de TI. Com isso, encontrar profissionais qualificados é essencial para que as empresas do ramo evitem custos desnecessários, melhorem a eficiência de suas atividades e criem vantagens competitivas.

% O Brasil é o segundo país com dificuldades para encontrar candidatos tecnicamente qualificados. Os investimentos em TI por parte das empresas e governo para atender a Copa do Mundo e os Jogos Olímpicos, 2014 e 2016, respectivamente, e os recentes incentivos fiscais do Governo sobre equipamentos de rede (incluindo dispositivos para o consumidor, como smartphones) contribuem para aumentar a lacuna de habilidades. 
 
%No entanto, antes de se tornar um profissional qualificado, um indivíduo precisa anteriormente ser um estudante. O início de uma carreira profissional está no processo de aprendizagem e formação. Nesta fase, precisarão ser conhecidas e desenvolvidas as habilidades necessárias para se tornar um profissional qualificado.

%A universidade tem um papel relevante nesse processo. Torna-se essencial que os educadores dos futuros profissionais conheçam os requisitos para formá-los, oferecendo uma educação que combina conhecimento do profissional e das necessidades do mercado de trabalho. Ou seja, é preciso saber identificar as características que compõem o perfil de profissional qualificado com o objetivo de guiar os aprendizes e ensiná-los a desenvolver as habilidades necessárias.

Podemos destacar dentro do setor de TI, empresas de desenvolvimento de software, que também apresentam uma alta demanda por profissionais \cite{kennan:09}. Um dos cargos nessa área é o de programador de software, atividade comumente atribuída ao início da carreira. O programador é responsável por traduzir o projeto do software em código-fonte. Nesse contexto, muitas habilidades técnicas são necessárias, tais como conhecimento em linguagens de programação e banco de dados.

No entanto, há uma crescente consciência de que apenas as habilidades técnicas não são suficientes para o sucesso no desenvolvimento de software, especialmente no mercado de trabalho de hoje, que é dinâmico, complexo e competitivo \cite{joseph:10}. Por isso, gestores e profissionais de recursos humanos também consideram as habilidades não-técnicas como fatores importantes na qualificação de um profissional.

Habilidades não-técnicas, conhecidas como soft skills, são características associadas à personalidade, como por exemplo, ser comunicativo, persistente, atencioso, etc. As soft skills guiam um indivíduo em suas ações, decisões e comportamento. Um profissional que busca desenvolver soft skills irá apresentar capacidade de influência, maior liderança e gestão de relacionamento \cite{hjyunus:12}.

Como as soft skills são relevante para qualificação, as empresas de desenvolvimento de software necessitam compreender quais delas são requeridas para os profissionais no desempenho de suas atividades. Principalmente durante o processo de contratação, elas precisam saber identificar entre os candidatos, quais dele possuem o conjunto de soft skills necessárias para cumprir os requisitos das vagas disponíveis.
% Quando se trata de contratação de profissionais, as empresas normalmente não conhecem os seus candidatos com antecedência. Portanto, é difícil identificar se eles têm as habilidades sociais necessárias para executar os empregos disponíveis.

No entanto, a identificação de soft skills é uma tarefa que consome tempo, pois é necessário conhecer o indivíduo e seu comportamento durante um determinado período, até ter condições de reconhecer suas habilidades. Algumas vezes também, requer esforços como entrevistas com os candidatos ou recomendações externas.
Quando o processo de identificação de soft skills falha, a empresa corre o risco de contratar pessoas que não se encaixam nos cargos disponíveis, gerando custos desnecessários e mal desempenho por parte do profissionais.
% Por outro lado, não identificando habilidades sociais pode levar a contratar alguém que não se encaixa no trabalho. Isso gera custos desnecessários para a empresa e pode levar o profissional a falhar.
Nesse contexto, a maioria das pessoas que falham no trabalho, não falham devido à falta de habilidades técnicas, mas sim devido a soft skills subdesenvolvidas \cite{bolton:86, mcgee:96}. %\cite{bolton:86}, \cite{mcgee:96}.

% No contexto de desenvolvimento de software, há alguns estudos que
Pesquisas anteriores reforçam a importância das soft skills \cite {ahmed:12, sterling:03, rehman:12},
%\cite {ahmed:12}, \cite{sterling:03}, \cite{rehman:12},
em geral, apenas indicando uma lista delas.
No entanto, as empresas precisam de mais informações
% Tal como conceitos
sobre o que as soft skills representam a respeito da personalidade de um indivíduo e como se aplicam no contexto do ambiente de trabalho.
Também falta uma abordagem automática capaz de identificar soft skills em indivíduos, oferecendo meios que facilitem esse processo.
%Assim, ainda precisamos de um estudo mais exploratório, que não apenas aponte soft skills importantes, mas também que trate de seus conceitos .


\section{Objetivos}

Para minimizar o problema da identificação de soft skills, nesta dissertação, propomos uma estratégia focada no papel do programador de software.
Propomos desenvolver maneiras de identificá-las automaticamente. Para isso, levamos em consideração alguns pontos.
%Visamos desenvolver maneiras de identificar as soft skills de um programador de software.
%em indivíduos que ainda estão em processo de formação profissional, ou seja, em estudantes de cursos superiores que estão aprendendo programação.

%Propomos que essa identificação seja feita de maneira automática, através da observação dos mesmos em um contexto virtual. Nossa estratégia, utiliza um ambiente de programação online (juiz online), que pode ser descrito como uma ferramenta que contém problemas de programação, os quais são resolvidos pelos usuários através soluções enviadas pelos mesmos e avaliadas pela ferramenta.

Primeiramente, sabemos que, para identificar as soft skills de um programador, é necessário observá-lo em suas atividades de programação. Além disso, como a identificação deve ocorrer automaticamente, precisamos utilizar um ambiente que possibilite a coleta automática de informações a respeito do comportamento de um programador.

Assim, nossa estratégia consiste no desenvolvimento de métricas que atribuem uma pontuação para cada soft skill de um programador. O ambiente em que podemos coletar essas métricas automaticamente é um juiz online, sistema no qual os usuários praticam atividades de programação. Esse tipo de sistema possui uma base de dados que guarda as interações de seus usuários, as quais podem ser analisadas a partir das referidas métricas, representando uma alternativa para identificação das soft skills.

Portanto, os objetivos desta dissertação incluem conhecer as soft skills importantes para o programador de software e desenvolver formas de identificá-las de maneira automática, através da aplicação de métricas, no contexto de um juiz online.
%, em estudantes de programação que são usuários de um juiz online.

\section{Metodologia}

Para atingir nossos objetivos, inicialmente é preciso fazer um levantamento de quais são as soft skills relevantes para um programador, compreender o que cada uma delas significa e como se aplicam no contexto da referida profissão. Esse conhecimento possibilita um embasamento teórico para a definição de métricas aplicadas no contexto de um juiz online, as quais servem para identificar se os indivíduos, usuários do sistema, possuem soft skills.

Assim, a metodologia de desenvolvimento deste trabalho inclui: 

\begin{enumerate}[label={(\roman*)}]
	\item O levantamento de soft skills importantes para profissionais de TI, com foco no papel de programador de software;
	\item O esclarecimento do conceito de cada soft skill do programador;
	\item A definição de métricas aplicadas no contexto de um juiz online para identificação automática de soft skills.
	\item A validação das métricas para identificação de soft skills.
\end{enumerate}

O levantamento das soft skills visa identificar os principais atributos profissionais que são requeridos no ramo de TI, com foco no programador de software. Nesse sentido, foi feita uma revisão bibliográfica que possibilitou uma visão geral das soft skills que as empresas buscam e consideram diferencial de sucesso para um programador e profissionais de TI.

Com base no levantamento bibliográfico, as soft skills que abordamos nesta dissertação são: Análise e resolução de problemas, Atenção a detalhes, Aprendizagem rápida, Persistência, Comunicação e Trabalho independente. 

Detalhamos essas soft skills em seus significados e aplicações, com o objetivo de entender melhor o que cada uma delas representa no ambiente de trabalho da profissão do programador e levantar informações suficientes para definir meios de identificá-las de maneira automática.

Para conceituar as soft skills, esta pesquisa visa associá-las a teorias da Psicologia que tratam a respeito de personalidade. Especificamente, propomos a associação das soft skills com os fatores de personalidade descritos pelo Modelo dos Cinco Fatores (Five Factor Model - FFM) \cite{mccrae:92}. 

Como dito anteriormente, esse conhecimento teórico é importante para subsidiar o desenvolvimento das métricas para identificação das soft skills, as quais aplicamos no contexto de um juiz online. Neste trabalho, utilizamos para esse fim o juiz online Huxley \cite{paes:13}, que pode ser acessado através do endereço eletrônico \href{thehuxley.com}{\textsl{thehuxley.com}}.

Para validação das métricas, conduzimos um estudo empírico com o objetivo de avaliar se as mesmas são capazes de identificar as soft skills que cada indivíduo possui.

\section{Relevância}

As contribuições deste trabalho estão voltadas para ampliar a compreensão dos aspectos humanos relacionados ao perfil do programador. Como qualquer outra atividade de desenvolvimento de software, a programação envolve elementos humanos que precisam ser considerados, afinal os produtos de software são desenvolvidos por pessoas e para pessoas \cite{john:05}.

Através do levantamento e conceituação das soft skills relevantes para papel do programador de software, esse profissional poderá entender o que o ambiente de trabalho exige, além de ser estimulado à reflexão sobre quais habilidades possui e quais precisam ser melhoradas.

Para estudantes de programação e seus educadores, também é aplicável. Baseando-se no fato de que o aprendizado é a base que constitui o profissional, é importante que ainda enquanto estudantes, as habilidades necessárias para o desempenho da função de programador sejam identificadas. Isso auxilia o aluno a conhecer em que será exigido no mercado de trabalho. E auxilia os educadores a incentivar os pontos fortes de seus educandos e guiá-los no desenvolvimento das habilidades que merecem mais atenção.

Essa identificação também é importante em estágios iniciais da carreira do profissional, pois é uma fonte para indicação dos mesmos à contratação em primeiro emprego ou estágios. Como estudantes, os indivíduos ainda não possuem experiência profissional. Para empresas em processo de contratação, resta apenas indicações da universidade e de seus educadores.

Além disso, as métricas propostas por este trabalho podem ser implantadas em um juiz online, como o Huxley. O ambiente de programação online que implementar essas métricas pode contar com uma nova funcionalidade de identificar soft skills em seus usuários. E as empresas podem utilizá-las com uma alternativa para facilitar o processo de contratação e identificação de candidatos qualificados.

%Até o momento, existem alguns estudos que reforçam a importância de soft skills para um profissional programador que, no geral, apenas indicam uma lista delas. No entanto, profissionais, estudantes, educadores e empresas de programação precisam de mais informações, conceitos sobre o significado de cada soft skill e o que exatamente elas representam quando se trata de personalidade e características de um indivíduo.

%Com isso, este trabalho se propõe a suprir a necessidade de um estudo mais exploratório, não só para citar as soft skills importantes para os programadores, mas também abordar os seus conceitos e definir formas de identificá-los em indivíduos.

\section{Validação}

Os resultados deste trabalho foram gerados através da aplicação das métricas para identificar soft skills em um grupo de usuários do Huxley. Para validação dos resultados foi conduzido um estudo empírico com o objetivo de avaliar se as métricas aqui propostas são capazes de identificar as soft skills que cada indivíduo possui.
Para tal validação, nos baseamos na relação que existe entre as soft skills e os fatores de personalidade. Propomos comparar se o resultado obtido através da aplicação das métricas identificam as soft skills associadas aos fatores de personalidade de um indivíduo.

Para conhecer a respeito da personalidade dos usuários que fazem parte de nosso estudo, os convidamos para responder um breve questionário sobre os fatores de personalidade descritos pelo FFM. Esse questionário é chamado de Ten-Item Personality Invetory (TIPI) \cite{gosling:03}.

Para comparar os resultados das métricas e do questionário de personalidade, estamos considerando o coeficiente de correlação de Pearson. Nossos resultados indicam que as métricas para identificação das soft skills análise e resolução de problemas, atenção aos detalhes, aprendizagem rápida e persistência são satisfatórios. Por outro lado, as métricas das soft skills comunicação e trabalho independente não apresentam os resultados desejados. 

Apesar de não ter sido possível validar todas as métricas, a maioria delas pode ser aplicada para identificação automática de soft skills. Além disso, outras contribuições deste trabalho incluem um melhor entendimento de quais são as soft skills importantes para o programador, trazendo informações para o profissional, no reconhecimento dos atributos que precisa desenvolver, para os educadores e aprendizes de programação, no processo de desenvolvimento de suas habilidades. Adicionalmente, empresas podem utilizar as métricas como alternativa para identificar os profissionais que precisam contratar. Trazendo ainda melhorias para Huxley ou qualquer juiz online que implante as métricas que propomos e validamos.

\section{Estrutura da dissertação}

Deste ponto em diante, o trabalho de dissertação está dividido nos seguintes capítulos:

\begin{itemize}

	\item \textbf{Capítulo \ref{chap:theorical}}: Traz a discussão teórica dos elementos que fundamentam esta pesquisa. A seção \ref{sec:sskills} explica o que são soft skills. A seção \ref{sec:ffm} apresenta a teoria de personalidade abordada pelo Modelo dos Cinco Fatores (FFM), o qual foi aplicado para expor o conceito de cada soft skill e para proceder com o estudo de validação.

	\item \textbf{Capítulo \ref{chap:research}}: Apresenta o levantamento das soft skills. Cada seção resume estudos que tratam da demanda por soft skills em profissionais de desenvolvimento de software. Focando no papel do programador, listamos quais soft skills adotamos nesta pesquisa.

	\item \textbf{Capítulo \ref{chap:concepts}}: Aborda a conceituação das soft skills que escolhemos no capítulo anterior. Em cada seção, explicamos o que significa possuir cada habilidade, por que ela é importante para o programador de software e como se aplica no contexto da profissão. Nesse capítulo também propomos, a associação que cada soft skill tem com os fatores de personalidade descritos pelo FFM.

	\item \textbf{Capítulo \ref{chap:metrics}}: Descreve a estratégia adotada para identificação automática das soft skills. Na seção \ref{sec:huxley}, apresentamos o juiz online Huxley e suas principais funcionalidades. Na seção \ref{sec:metrics}, propomos sete métricas para identificação das soft skills. 

	\item \textbf{Capítulo \ref{chap:evaluation}}: Descreve o estudo de validação. A seção \ref{sec:participantes} traz informações a respeito dos participantes. A seção \ref{sec:material} apresenta o material utilizado no estudo. E a seção \ref{sec:procedimento} explica o procedimento de aplicação do questionário de personalidade.

	\item \textbf{Capítulo \ref{chap:results}}: Trata dos resultados. Apresentamos, na seção \ref{sec:tipitipi}, as correlações entre os itens do questionário TIPI, para analisar a consistência das respostas dos participantes. Já na seção \ref{sec:tipiss}, apresentamos as correlações entre os itens do questionário TIPI e as métricas das soft skills. Discutimos os resultados na seção \ref{sec:discussao}.

	\item \textbf{Capítulo \ref{chap:conclusion}}: Conclui este trabalho de dissertação, discutindo os resultados encontrados e também indicando trabalhos futuros para continuação e aperfeiçoamento desta pesquisa.
	
\end{itemize}

