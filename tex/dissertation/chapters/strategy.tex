
%5%%5%%5%%5%%5%%5%%5%%5%%5%%5%%5%%5%%5%%5%%5%%5%%5%%5%%5
%                     Estratégia                       %
%5%%5%%5%%5%%5%%5%%5%%5%%5%%5%%5%%5%%5%%5%%5%%5%%5%%5%%5

\chapter{Estratégia para identificação automática de soft skills}

\label{chap:strategy}

Conhecemos a importância das soft skills para o processo de contração na Subseção \ref{ss-imortance}. Nesse contexto, é comum que as empresas encontrem dificuldades para identificar habilidades de seus candidatos. Isso porque identificar soft skills é uma tarefa que consome tempo, pois é necessário conhecer o indivíduo e seu comportamento durante um determinado período, até ter condições de reconhecer suas habilidades. Para minimizar esse problema, apresentamos uma estratégia que foca no programador de software e visa identificar soft skills de maneira automática.

Dividimos a estratégia para identificaçao automática de soft skills em três passos: (1) Levantamento de soft skills, (2) Conceituação das soft skills, (3) Desenvolvimento de métricas para identificação das soft skills. Podemos observar esses passos na Figura \ref{img:strategy}.

A estratégia é baseada em, inicialmente, entender o significado das soft skills, para então buscar formas de identificá-las.
Durante o primeiro passo, fizemos o levantamento de algumas soft skills do referido papel, tema abordado no Capítulo \ref{chap:research}
O próximo passo é dado no capítulo \ref{chap:concepts}, onde formamos uma base teórica e conceitual a respeito das soft skills. Esse conhecimento é importante para subsidiar o desenvolvimento de métricas para identificação automática.

Abordamos no capítulo \ref{chap:metrics}, o terceiro passo da estratégia, ou seja, o desenvolvimento das métricas para identificação. 

